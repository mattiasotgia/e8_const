\documentclass[a4paper, varvw, nofootinbib]{revtex4-2}
\usepackage[a4paper,top=1.05in, bottom=0.85in, left=0.85in, right=0.75in]{geometry}
\usepackage[italian]{babel}
\usepackage[utf8]{inputenc}
\usepackage[T1]{fontenc}
\usepackage{microtype}
\usepackage{newtxtext,newtxmath}
\makeatletter
\let\it@comma@def\active@comma
\makeatother
\usepackage{physics}
\usepackage[round-mode = uncertainty]{siunitx}
\usepackage{graphicx}
\usepackage[hidelinks]{hyperref}

\def\bibsection{\section*{\refname}}

\begin{document}
\title{Determinazione della carica elementare, della costante di Plank e della costante di Boltzmann}
\author{Andrea Parodi}
\author{Francesco Polleri}
\author{Mattia Sotgia}
\email{s4942225@studenti.unige.it}
\affiliation{Dipartimento di Fisica, Università degli Studi di Genova, Genova, Italia}
\date{7 dicembre 2022}

\begin{abstract}
Presentiamo la misura dei valori della carica dell'elettrone $e$, della costante di Planck $h$ e della costante di Boltzmann $k$ come combinazione di quattro esperimenti separati ed indipendenti, mirati alla misura dei rapporti $e/h$, $e/k$ \cite{inmanMeasurementIntroductoryPhysics1973} e $h/k$, insieme all'esperimento proposto da Millikan \cite{millikanIsolationIonPrecision1911} per ottenere un valore della carica elementare $e$. Ci concentriamo sulla misura del rapporto $h/k$ realizzando lo stesso esperimento proposto in \cite{crandallMinimalApparatusDetermination1983}, sfruttando quindi la radiazione di corpo nero in una sua applicazione realizzabile in laboratorio. Combiniamo quindi i risultati in uno studio statistico ottenendo un valore preciso per le costanti $h$, $e$, $k$. 
\end{abstract}

\maketitle

\subparagraph*{Introduzione} Vogliamo ottenere una misura precisa del valore della carica dell'elettrone $e$, della costante di Planck $h$ e della costante di Boltzmann $k$. Per ottenere la misura di queste tre costanti ci concentriamo sulla realizzazione di quattro esperimenti, separati ed indipendenti. Non è facile infatti, eccetto per il caso della carica elementare, realizzare un singolo esperimento che permetta di misurare il valore di $h$, $k$ in modo diretto, ma sono invece realizzabili alcuni esperimenti che permettono di ottenere il valore dei rapporti a coppie di queste tre quantità \cite{inmanMeasurementIntroductoryPhysics1973, millikanIsolationIonPrecision1911, crandallMinimalApparatusDetermination1983}. Otterremo infatti la misura del rapporto $e/h$ studiando la curva di risposta del passaggio di corrente attraverso una giunzione \emph{p-n}, come un {LED}, rispetto alla tensione fornita in alimentazione. La misura di $e/k$ è invece ottenuta in modo simile, però studiando la risposta di un transistor ad un segnale di tensione in ingresso. Infine a partire dalla legge di Boltzmann, sfruttando la legge della radiazione di corpo nero di Planck andiamo a calcolare il valore di $h/k$. 

Ai tre esperimenti realizzati in laboratorio sono poi aggiunti dati di misurazioni dell'esperimento di Millikan che permettono di ottenere una stima di $e$. Dati quindi i quattro valori ottenuti possiamo ottenere una miglior stima delle costanti fisiche realizzando una analisi statistica.

\paragraph*{Risposta in corrente di una giunzione \emph{p-n}, misura di $e/k$}

È noto per le conclusioni tratte già da \cite{einsteinConcerningHeuristicPoint1965} e poi dai risultati della meccanica quantistica che una particella, carica elettricamente, se colpita con sufficiente energia $E$ produce un fotone di frequenza $\nu$ e una particella identica ma di energia inferiore. La relazione che lega $E$ e $\nu$ è data da $E=h\nu$, e la costante di proporzionalità è la costante di Planck. Non è però facile misurare in modo diretto il valore dell'energia per poter ricavare una stima immediata di tale costante. In un diodo, ovvero un elemento circuitale costituito da due giunzioni semiconduttrici drogate in modo diverso, avremo che il passaggio di corrente avviene quando gli elettroni della giunzione $n$ sono spinti a uscire dalle loro orbite stabili e andare a rimediare alla mancanza di cariche negative della giunzione $p$. Quando questo passaggio avviene gli elettroni rilasciano energia, in forma quantizzata, come fotoni di luce. Nei led il drogaggio \emph{p-n} è tale per cui l'energia rilasciata $E$ corrisponde una frequenza $\nu$ nel visibile. L'energia che viene rilasciata è pari al lavoro che viene compiuto per spostare gli elettroni, che in termini della tensione si traduce in un lavoro $\mathcal W = e V_\text{diode}$, che permette di stabilire una relazione lineare $h\nu = eV_\text{diode}$ per cui trovato il valore di tensione per cui si attiva il LED, si può ottenere una stima di $e/h$.

\paragraph*{Risposta in tensione di un transistor, misura di $e/h$}

Come i LED anche i transistor sfruttano giunzioni \emph{p-n} di semiconduttori. Quando una giunzione \emph{p-n} viene creata, la corrente che si instaura è esponenzialmente proporzionale alla tensione, $I=A\exp(-eV_0/kT)$, con un coefficiente che permette di trovare una stima di $e/k$. Però la relazione in condizioni reali deve essere corretta, dotandosi di un coefficiente $\eta$ che varia per ogni diodo. Stabilità maggiore è però ottenuta considerando una giunzione tripla \emph{n-p-n}, mantenendo la prima e l'ultima giunzione, ovvero emettitore e collettore, allo stesso potenziale. In questo caso la corrente elettrica è descritta come \begin{equation} I = I_0 \qty(\exp(\frac{eV_{t}}{kT}) - 1). \end{equation}

\paragraph*{Radiazione di corpo nero} Per un corpo nero è valida la relazione trovata da Planck per cui l'intensità di radiazione emessa ha una dipendenza dalla temperatura descritta dalla relazione \begin{equation} \mathcal I(\nu, T) = \frac{8 \pi h \nu^3}{c^2}\qty(\exp(\frac{h\nu}{kT})-1)^{-1}.  \end{equation} Selezionando una sola frequenza di radiazione $\nu$, allora avremo che dividendo membro a membro otteniamo una relazione più agevole e siccome per frequenze nel visibile ($\SI{350}{\nano\metre} < \nu <\SI{700}{\nano\metre}$) possiamo considerare che $h\nu \ll kT$, ovvero che l'esponenziale si sviluppa intorno allo zero, possiamo quindi semplificare la relazione precedente che diventa \begin{equation} \log(\frac{\mathcal I_j}{\mathcal I_i}) = \frac{h\nu}{k}\qty({1\over T_i} - {1\over T_j}),~\text{ dove $h/k$ è considerato parametro di fit.} \label{eq:3}\end{equation}

In una prima approssimazione una lampadina alogena può essere considerata essere un esempio di corpo nero, per cui il bilancio di potenza fornisce $P_\text{abs} + P_e = P_\text{em} + P_\text{cond}$, dove $P_\text{amb} = AT_\text{amb}^4$ è la potenza assorbita dall'ambiente, che consideriamo trascurabile, come anche $P_\text{cond}$ che è legata alla conduzione elettrica. 
Vogliamo ora osservare eventuale dipendenza della temperatura di emissione dalla temperatura ambiente per il setup considerato. Da prima avremo che $P = AT^4$. Assumendo allora una dipendenza della temperatura dalla resistenza come $T = \beta R^\gamma$, possiamo anche ottenere una relazione per la dispersione in ambiente, per cui $P_\text{amb} = AT_\text{amb}^4 = \beta A R^{4\gamma}_\text{amb}$, e per la potenza emessa dal corpo nero, per cui $P = AT^4 = \beta A R^{4\gamma}$. Dividendo membro a membro avremo allora la dipendenza della temperatura dalla temperatura ambiente, e sostituendo per la potenza emessa avremo dunque \begin{equation} P = VI =  A \qty[\qty(\frac{V}{IR_\text{amb}})^\gamma T_\text{amb}]^4.\label{eq:4} \end{equation} 

\subparagraph*{Metodi}\label{sec:black_body_methods} Il setup sperimentale piuttosto semplificato prevede una lampada alogena che emette radiazione non polarizzata su uno spettro ampio. Di questo spettro si considera un taglio preciso grazie ad un filtro ottico che permette di avere $\nu = \SI{6.66(6)e+5}{\giga\hertz}$ e quindi lunghezza d'onda $\lambda \simeq \SI{450}{\nano\metre}$. L'intensità luminosa viene poi quantificata utilizzando un fotodiodo per il quale $V_\text{fd} \propto \mathcal I_\text{fd}$, per cui la relazione \eqref{eq:3} si ottiene nella forma \begin{equation}\underbrace{\log(\frac{V^\text{fd}_j}{V^\text{fd}_i})}_{(\text{LOG})_i^j} = \frac{h}{k}\underbrace{\nu\qty({1\over T_i} - {1\over T_j})}_{{\Delta_{\frac{1}{T}}}_i^j}. \label{eq:5}\end{equation}

\begin{figure}
    \centering
    \includegraphics[width=14cm]{fig/plot_hk}
    \caption{\textbf{a.} Considerati coppie di punti in tensione e corrente, possiamo caratterizzare il circuito della sorgente e la sua risposta in termini di temperatura, ovvero ottenere il valore di $\gamma$. \textbf{b.} I dati raccolti e trasformati secondo la relazione \eqref{eq:5} sono fittati su un set di dati considerato $j=0$.}
\end{figure}

\paragraph*{Misura della dipendenza da $\gamma$ per la temperatura} Caratterizziamo l'apparato sulla base della relazione \eqref{eq:4}, ovvero otteniamo un valore di $\gamma$ per cui la relazione è valida. Ipotizzando la temperatura ambiente costante, avremo che, a meno di costanti, la relazione \eqref{eq:4} ci permette di avere una stima di $\gamma$. Troviamo allora che \[\gamma = \num{0.87769(24)}.\] Noto quindi il valore di $\gamma$ è possibile determinare la temperatura della sorgente in funzione della tensione fornita e della corrente che circola sul circuito. La temperatura risulterà essere allora una funzione $T_j = T_j(V_j, I_j, \gamma)$, oltre che dipendente dalla temperatura ambiente $T_\text{amb}$ (misurata indipendentemente durante la presa dati, $T_\text{amb} = \SI{299.7 +- 0.7}{\kelvin}$) e dall'impedenza a temperatura ambiente $R_\text{amb} = \SI{25.045 +- 0.215}{\ohm}$ misurata a priori della presa dati. Osserviamo dunque che quindi i diversi valori di temperatura $T_j$ risulteranno essere tra loro correlati. 

\paragraph*{Calcolo del rapporto $h/k$} La relazione \eqref{eq:5} è una relazione che permette facilmente di ottenere una buona stima del rapporto $h/k$, a patto di tenere in considerazione tutte le possibili correlazioni che naturalmente intervengono nel passaggio dal set di dati $(V_j^\text{fd}, T_j)$ alla forma $((\text{LOG})_i^j, ({\Delta_{\frac{1}{T}}}_i^j))$ dove queste ultime notazioni indicano i due membri della relazione lineare espressa in \eqref{eq:5}. Possiamo però ipotizzare di non considerare gli errori sulle temperature, e associare alla funzione $(\text{LOG})_i^j$ solo l'errore legato al valore di $V_j^\text{fd}$. 
Inoltre, dopo aver appurato che il metodo e il risultato non dipendano da tale scelta, possiamo da ora in poi utilizzare la convenzione per cui nella relazione \eqref{eq:5} poniamo $i=0$. Quindi si potrà eseguire un fit con le variabili $(\text{LOG})_0^j\pm\sigma_\text{LOG$_0^j$}$ e ${\Delta_{\frac{1}{T}}}_0^j$. A questo punto, facendo variare individualmente ogni valore che determina il fit della funzione, consideriamo lo scarto legato ad ognuna variazione individuale, e, dopo aver verificato che in questo modo lo scarto sia inferiore percentualmente all'errore percentuale del fit iniziale, procediamo a sommare in quadratura tutti i contributi, fino ad ottenere il valore della deviazione standard associata alla miglior stima di $h/k$. Otteniamo allora \[h/k = \SI{3.27(4)e-11}{\kelvin\second}.\]

\begin{figure}
\centering
\includegraphics[width=14cm]{fig/millikan}
\caption{Minimizzazione della Eq. \eqref{eq:6} con \emph{maximum Likelihood (MLE)} per ottenere una stima dei parametri della distribuzione di probabilità $f_\text{Mlk}$. Il parametro principale che interessa ottenere è però il valore di separazione tra i centri delle distribuzioni gaussiane, che corrisponde alla carica dell'elettrone $e$.}\label{fig:millikan}
\end{figure}

\subparagraph*{Determinazione indipendente della carica elementare}\label{sec:millikan} Utilizzando un set di dati dall'esperimento di Millikan vogliamo ottenere una stima indipendente della carica elementare $e$. L'esperimento ci permette di avere il valore della carica elementare sfruttando proprio il fenomeno della quantizzazione. Infatti una differenza di potenziale si pone in equilibrio con la forza gravitazionale, permettendo di conoscere la carica depositata su gocce d'olio. Queste si posizioneranno a distanza proporzionale alla carica elementare. I dati raccolti saranno adattabili a più gaussiane, separate ciascuna da un delta $\propto e$. La forma della distribuzione di probabilità sarà quindi trovata come \begin{equation} f_\text{Mlk} =  A_0 \frac{1}{\sqrt{2\pi\sigma^2}}\exp({-\frac{e}{2\sigma^2}}) +  A_1 \frac{1}{\sqrt{2\pi\sigma^2}}\exp({-\frac{2e}{2\sigma^2}}) + (1-A_0-A_1) \frac{1}{\sqrt{2\pi\sigma^2}}\exp({-\frac{3e}{2\sigma^2}}), \label{eq:6}\end{equation} dove i fattori $A_0$, $A_1$ e $1-A_0-A_1$ sono necessari per rendere di norma unitaria la distribuzione. Eseguendo una massimizzazione \emph{binned} della Likelihood (minimizziamo $-\log \mathcal L(f_\text{Mlk})$) otteniamo allora la stima della carica elementare \[e(\text{Mlk}) = \SI{1.608(20)e-19}{\coulomb}.\] Come possiamo anche osservare in fig. \ref{fig:millikan} i tre picchi gaussiani sono limitati ad una regione per cui la carica in unità elementari è $<\SI{5.8}{\elementarycharge}$. Dopo questo valore abbiamo che i singoli picchi gaussiani non sono distinguibili, e che comunque i dati sono compatibili con lo zero. La minimizzazione è quindi eseguita imponendo un limite superiore per la carica \SI{5.8}{\elementarycharge}.

\begin{table}
    \caption{Riportiamo il risultato degli esperimenti realizzati per misurare le costanti $e$, $h$, $k$.}\label{tab:results_indip}
    \begin{ruledtabular}
        \begin{tabular}{lrr}
            Quantità & Valore sperimentale & Valore Teorico\\\colrule
            Carica elementare da esperimento di Millikan ($\times\SI{e-19}{\coulomb}$) & \num{1.608 +- 0.020}&\num{1.602176634}\\
            Costante di Planck su costante di Boltzmann $h/k$ ($\times\SI{e-11}{\kelvin\second}$) & \num{3.27 +- 0.04} & $\num{6.62607015}/\num{1.380649}\simeq\num{4.799}$\\
            Carica elementare su costante di Boltzmann $e/k$ ($\times\SI{e4}{\coulomb\kelvin\per\joule}$) & \num{1.1440 +- 0.0040} & $\num{1.602176634}/\num{1.380649}\simeq\num{1.1605}$\\
            Carica elementare su costante di Plank\footnote{Il valore considerato è una scelta arbitraria tra due possibili valori, entrambi riportati, non si riportano i calcoli per l'altro valore. Si sceglie il valore $h/k = \SI{0.252 +- 0.010 e15}{\coulomb\per\joule\second}$.} $e/h$ ($\times\SI{e15}{\coulomb\per\joule\second}$) & \num{0.252 +- 0.010} & $\num{1.602176634} / \num{6.62607015} \simeq \num{0.242}$\\
            & \num{0.201 +- 0.006} & $\num{1.602176634} / \num{6.62607015} \simeq \num{0.242}$
        \end{tabular}
    \end{ruledtabular}
\end{table}

\subparagraph*{Calcolo delle costanti $e$, $h$, $k$ e conclusioni}\label{sec:combined_data} Abbiamo un set di dati costituito da risultati indipendenti, come riportati in tabella \ref{tab:results_indip}. Volendo stimare i valori delle costanti dai dati che abbiamo considerato, allora possiamo pensare di minimizzare il valore del $\chi^2$ costruito come \begin{equation} 
    \chi^2_{e,h,k} = 
    \qty(\frac{\mu_e(\text{Mlk}) - e_\text{exp}}{\sigma_e(\text{Mlk})})^2 + 
    \qty(\frac{\mu_{h/k} - h_\text{exp}/k_\text{exp}}{\sigma_{h/k}})^2 + 
    \qty(\frac{\mu_{e/k} - e_\text{exp}/k_\text{exp}}{\sigma_{e/k}})^2 + 
    \qty(\frac{\mu_{e/h} - e_\text{exp}/h_\text{exp}}{\sigma_{e/h}})^2, \label{eq:chi2_ehk}
\end{equation} sui parametri $e_\text{exp}$, $h_\text{exp}$ e $k_\text{exp}$.

In questo modo, accertato che le misure sono indipendenti e che quindi la stima della deviazione standard sul risultato del fit è corretta, otteniamo, al $68\%$ di probabilità dei valori delle costanti come \[
    \begin{aligned}
        e_\text{exp} &= \SI{1.608 +- 0.020 e-19}{\coulomb}\text{, compatibile con la teoria al $68\%$},\\
        h_\text{exp} &= \SI{4.80 +- 0.08 e-34}{\metre\squared\kilo\gram\per\second}\text{, non compatibile alla teoria neanche al 99.7\%},\\
        k_\text{exp} &= \SI{1.410 +- 0.018 e-23}{\metre\squared\kilo\gram\per\second\squared\kelvin}\text{, non compatibile con la teoria.}.
    \end{aligned}
\] Per il valore di $k_\text{exp}$ in particolare osserviamo che però probabilmente la compatibilità è entro $4$-$5\sigma$, mentre per $h_\text{exp}$ per avere compatibilità dovremmo considerare un numero ancora maggiore di $\sigma$. 

<<<<<<< HEAD
Un motivo per il quale abbiamo ottenuto un valore di $h_\text{exp}$ particolarmente lontano dal valore teorico potrebbe essere legato al fatto che durante la presa dati per l'esperienza del corpo nero, non abbiamo atteso un tempo sufficientemente lungo tra le varie misure affinchè la temperatura della lampadina potesse stabilizzarsi. Questo ha comportato il fatto che la lampadina, essendo ad una temperatura più bassa del dovuto in relazione alla corrente che in quel determinato istante stavamo fornendo, ha emesso radiazione in uno spettro diverso da quello atteso.

Un altro problema è legato al valore di $R_\text{amb}$. Infatti misurando tale valore ci siamo resi conto che il valore non era stabile e non seguiva un preciso andamento, ma continuava ad oscillare irregolarmente all'interno di un certo intervallo di valori. Non è valida quindi l'ipotesi più semplice per cui una possibile causa di instabilità potesse essere il raffreddarsi o scaldarsi della lampadina. C'è quindi la possibilità che il valore che abbiamo utilizzato per l'analisi dati non corrisponda a quello reale.
=======
Un motivo per il quale abbiamo ottenuto un valore di $h_\text{exp}$ particolarmente lontano dal valore teorico potrebbe essere legato al fatto che durante la presa dati per l'esperienza del corpo nero, non abbiamo atteso un tempo sufficientemente lungo tra le varie misure affinché la temperatura della lampadina potesse stabilizzarsi. Questo ha comportato il fatto che la lampadina, essendo ad una temperatura più bassa del dovuto in relazione alla corrente che in quel determinato istante stavamo fornendo, ha emesso radiazione in uno spettro diverso da quello atteso.

Un altro problema è legato al valore di $R_\text{amb}$. Infatti misurando tale valore ci siamo resi conto che il valore non era stabile e non seguiva un preciso andamento, ma continuava ad oscillare irregolarmente all'interno di un certo intervallo di valori. Non è valida quindi l'ipotesi più semplice per cui una possibile causa di instabilità potesse essere il raffreddarsi o scaldarsi della lampadina. C'è quindi la possibilità che il valore che abbiamo utilizzato per l'analisi dati non corrisponda a quello reale. 
>>>>>>> adfec12bcee7989fcda6e0a1eabd1abccf86bcf1

\begin{figure}
\centering
\includegraphics[width=15cm]{fig/ehk_contour_plot_68}
\caption{Intervalli di probabilità del 68\% ($\sigma$), del 95\% ($2\sigma$) e 99.7\% ($3\sigma$) ottenuti dalle curve di livello del $\chi^2_{e,h,k}$ in \eqref{eq:chi2_ehk}. Gli intervalli per distribuzioni bidimensionali sono ottenute considerando una curva per cui una deviazione standard corrisponde ad un taglio del $\chi^2_{e,h,k}$ a $+2.3$ rispetto al minimo, due deviazioni standard corrispondono a $+6.17$ dal minimo e tre deviazioni standard a $+11.8$ dal minimo. Le linee tratteggiate indicano i valori stimati sperimentalmente $e_\text{exp}$, $h_\text{exp}$ e $k_\text{exp}$, mentre con il simbolo $\star$ e con le linee continue rosse sono indicati i valori teorici.}\label{fig:3}
\end{figure}



\bibliography{ref/ref1}

\end{document}
